\documentclass[a4paper,addpoints]{exam}

\usepackage{amsmath,amssymb,amsthm}
\usepackage{hyperref}
\usepackage{titling}

\runningheader{CS/MATH 113}{WC15: Graphs}{\theauthor}
\runningheadrule
\runningfootrule
\runningfooter{}{Page \thepage\ of \numpages}{}

\printanswers

\title{Weekly Challenge 15: Graphs\\CS/MATH 113 Discrete Mathematics}
\author{team-name}  % <== for grading, replace with your team name, e.g. q1-team-420
\date{Habib University | Spring 2023}

\qformat{{\large\bf \thequestion. \thequestiontitle}\hfill}

\begin{document}
\maketitle
\thispagestyle{empty}

\begin{questions}

\titledquestion{Graphic sequence}

  The \textit{degree sequence} of a graph is the sequence of the degrees of the vertices of the graph in nonincreasing order. For example, the degree sequence of the graph $G$ in Example 1 of Section 10.2 in our textbook is $4,4,4,3,2,1,0$.

  A sequence $d_1 , d_2 , \ldots , d_n$ is called \textit{graphic} if it is the degree sequence of a simple graph.

  Determine whether each of the following sequences is graphic. For those that are, \href{https://www.baeldung.com/cs/latex-drawing-graphs}{draw a graph} having the given degree sequence. For those that are not, explain why they are not.
  \begin{parts}
  \part[2] $2,2,2,2,2$
  \part[2] $3,3,3,2,2$
  \part[2] $3,3,2,2,2$
  \part[2] $2,1,1,1,1$
  \part[2] $4,4,3,2,1$
  \end{parts}

  \begin{solution}
    
  \end{solution}
  
\end{questions}


\end{document}

%%% Local Variables:
%%% mode: latex
%%% TeX-master: t
%%% End:
